\documentclass[letterpaper,11pt,notitlepage]{article} 
\usepackage[left=2cm,top=2cm,right=2cm,bottom=2cm]{geometry}
% para poder escribir con tildes
\usepackage[T1]{fontenc}
\usepackage[utf8]{inputenc}
\usepackage[spanish]{babel}

% fuentes para escribir símbolos
\usepackage{amsfonts}
\usepackage{amssymb}
\usepackage{amsthm}
\usepackage{mathrsfs}

% inclusión de graficos
\usepackage{graphicx}

% \usepackage{hyperref}
\pagestyle{empty}
\usepackage[centertags]{amsmath}
% Codigo
\usepackage{listings}
% Varias Imagenes en una Figura
\usepackage{subfig}
% //posicion imagen
\usepackage{float}
\usepackage{subfig}
\usepackage[colorlinks=true,urlcolor=blue]{hyperref}
% \topmargin 0 


\usepackage{color}
\definecolor{gray97}{gray}{1}
\definecolor{gray75}{gray}{.75}
\definecolor{gray45}{gray}{.45}
\lstset{ frame=Ltb,
framerule=0pt,
aboveskip=0.5cm,
framextopmargin=3pt,
framexbottommargin=3pt,
framexleftmargin=0.4cm,
framesep=0pt,
rulesep=.4pt,
backgroundcolor=\color{gray97},
rulesepcolor=\color{black},
%
stringstyle=\ttfamily,
showstringspaces = false,
basicstyle=\small\small\ttfamily,
commentstyle=\color{gray45},
keywordstyle=\bfseries,
%
numbers=left,
numbersep=12pt,
numberstyle=\tiny,
numberfirstline = false,
breaklines=true,
}
% minimizar fragmentado de listados
\lstnewenvironment{listing}[1][]
{\lstset{#1}\pagebreak[0]}{\pagebreak[0]}
\lstdefinestyle{consola}
{basicstyle=\scriptsize\bf\ttfamily,
backgroundcolor=\color{gray75},
}
\lstdefinestyle{C}
{language=C,
basicstyle=\ttfamily
}

% ====================================

% % ===== Encabezado =====
% \pagestyle{myheadings}
% \markright{Tutorial Creación de proyecto en MPLABX \hfill }


% ===== Ajuste layout pagina =====
%\usepackage{fullpage}

% ================================

% --- commandos ---
% \newcommand{\ds}{\displaystyle}
% \def\x{{\bf x}}
% -----------------

% ========  Aca comienza el cuerpo del texto ==========
\begin{document}

% \maketitle

\begin{center}
\textsc{ \huge \bfseries Arquitectura de Computadores\\[0.2cm] 543.426}\\[0.2cm]
\textsc{ Ayudante: Antonio Saavedra}\\[0.2cm]
Ayudantía de pipeline\\

\today

\end{center}



\section{Problema 1}

Suponga un procesador MIPS segmentado en un pipeline de 8 etapas: 2 de fetch, 1 de decodificación, 2 de ejecución, 2 de acceso a memoria de datos y una de escritura en registros. El pipeline está regularizado, utiliza saltos retrasados que se resuelven en la etapa de decodificación, y soporta forwarding de datos en todas las etapas. Considere el siguiente código:
\begin{lstlisting}[style=C]
I1: lw   $t1, 0($a0)
I2: lw   $t2, 1000($a0)
I3: add  $t1, $t1, $t2
I4: sw   $t1, 2000($a0)
I5: addi $a0, $a0, 4
I6: bne  $a0, $a1, I1
I7: nop
I8: nop
\end{lstlisting}
\textbf{a)} Identifique las dependencia RAW, WAR, y WAW dentro de una ejecución de este código.
\\

\textbf{b)} Grafique la ejecución del código y calcule su CPI sostenido.
\\

\textbf{c)} Calcule el CPI sostenido graficando luego de reordenar el código para minimizar burbujas. Renombre registros y ajuste constantes de ser necesario.
\\

\textbf{d)} Calcule el CPI sostenido graficando luego de aplicar loop unrolling x2, renombrar registros y reordenar el código para optimizar su desempeño. Suponga que el lado itera un número par de veces.
\newpage
\section{Problema 2}

En un procesador MIPS con un pipeline en 12 etapas se pretende ejecutar el siguiente código:

\begin{lstlisting}[style=C]
 I1: lw   $t1, 0($a0)
 I2: add  $t1, $t1, $t0
 I3: lw   $t2, 40($a0)
 I4: add  $t2, $t2, $t0
 I5: mul  $t2, $t1, $t2
 I6: lw   $t1, 0($a1)
 I7: sub  $t1, $t2, $t1
 I8: add  $t2, $t1, $t2
 I9: sw   $t1, 4($a1)
I10: addi $a1, $a1, 4
I11: add  $a0, $a0, $t2
I12: beq  $a1, $t5, I1

 
\end{lstlisting}

Las etapas del pipeline se descomponen en 2 etapas de fetch, 2 etapas de decodificación, 3 de ejecución en ALU, 4 de acceso a memoria de datos y 1 de escrita en los registros. El pipeline está regularizado y ejecuta los saltos utilizando un predictor de saltos que se asume perfecto y los resuelve al finalizar el segundo ciclo de la etapa de decodificación. El pipeline soporta forwarding en todas sus etapas.
\\

\textbf{a)} Identifique las dependencia RAW, WAR, y WAW dentro de una ejecución de este código.
\\

\textbf{b)} Calcule el CPI sostenido graficando luego de reordenar el código para minimizar burbujas. Renombre registros y ajuste constantes de ser necesario.

\end{document}
